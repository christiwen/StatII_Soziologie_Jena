% \documentclass[a4paper,12pt]{article}
% \usepackage[utf8]{inputenc}
% \usepackage[ngerman]{babel} 
% %opening
% \title{}
% %\author{}
% \date{}
\documentclass[a4paper,fontsize=14pt]{article}
%\documentclass[a4paper,fontsize=13pt]{scrartcl}
%\definecolor{mygreen}{cmyk}{0.82,0.11,1,0.25}
\usepackage[utf8]{inputenc}
\usepackage{amsmath}
\usepackage{mathtools}
\usepackage{hyperref}
\usepackage[ngerman]{babel} 
\usepackage{textcomp} 
\usepackage{color}
\usepackage{xcolor}
\usepackage{multirow,array}
%\usepackage[margin=1.65cm
%  %,showframe% <- only to show the page layout
%]{geometry}
\usepackage{rotating}
\usepackage{booktabs} 
\usepackage{lscape}
\usepackage{float}
%opening
\title{Lehrplanung: Lineare und logistische Regression mit STATA WS 17/18}
%\subtitle{Simulation und Auswertung der simulierten Daten in Übung und Tutorium mit Hilfe von \texttt{R}}
%\author{Mariana Nold}
\date{\vspace{-10ex}}
\begin{document}

\maketitle

%Termine: 
%16./23.10.17
%13./27.11.17
%11.12.17
%08.01.18
%22.01.18

\section{Schwerpunkte}
\begin{enumerate}
\item{\colorbox{yellow!60}{Statistik }
\begin{itemize}
\item{Welcher Logik folgt das Schätzen und Testen?}
\item{Probleme statistische Tests und Auswege}
\item{Was ist  Regression?}
\item{Was ist (logistische) Regression?}
\item{Ein (un)gelöstes Problem: Die verständliche Interpretation der Parameter im Logit-Modell} %S. 567
\end{itemize}}
\item{\colorbox{pink}{\texttt{STATA}}
\begin{itemize}
\item{Daten mit \texttt{STATA} aufbereiten und beschreiben}
\item{Bekannte Tests mit \texttt{STATA} rechnen}
\item{Bivariate Analyse: Kreuztabellen, Vergleich von stetiger Variable in zwei Gruppen, Regression mit nur einer unabhängigen Variable}
\item{Regression mit einer oder  mehren unabhängigen Variablen}
\item{Logistische Regression mit einer oder  mehren unabhängigen Variablen}
\end{itemize}
}
\item{\colorbox{green!50}{Artikel}
\begin{itemize}
\item[1)]{Gesundheitsbezogene Lebensqualität von übergewichtigen 
und adipösen Jugendlichen-Welche Unterschiede zeigen sich nach  Sozialstatus und
Schulbildung ( Laura Krause, Ute Ellert, Lars Eric Kroll, Thomas Lampert )\\
\url{edoc.rki.de/oa/articles/reBtAzsyxON1Q/PDF/27nYdA0QWejZo.pdf}}
%\item[2)]{Psychische Gesundheit von übergewichtigen
%und adipösen Jugendlichen unter Berücksichtigung von Sozialstatus und
%Schulbildung. (Krause, L.)}
\item[2)]{ Bildungsbenachteiligung durch Übergewicht: Warum adipöse Kinder in der Schule schlechter abschneiden. (Marcel Helbig, Stefanie Jähnen)\\
\url{www.degruyter.com/view/j/zfsoz.2013.42.issue-5/zfsoz-2013-0504/zfsoz-2013-0504.xml}}
\end{itemize}
}
\end{enumerate}

\section{Statistische Methoden}
Zielkompetenzen, die geübt werden sollen sind:
\begin{enumerate}
\item{Statistische Ergebnisse verständlich kommunizieren, wesentliche Aussagen in einfachen Worten ausdrücken können.}
\item{Daten selbst analysieren: eigene Analysen, passend zu den Artikeln, wobei die Modelle in den Artikeln selbst nicht repliziert werden.}
\item{Statistische Informationen nutzen und sachadäquat interpretieren: STATA-Output und Information aus den bedien Artikeln.}
\end{enumerate}

\section{Literatur}

\begin{enumerate}
\item{Statistik:  \textit{Regressionsmodelle für Zustände und Ereignisse, Michael Windzio},  Kap 1- 3 (Thulb E-Book)}
\item{Statistik: \textit{Logistische Regression, Best, Henning (et al.),Handbuch der sozialwissenschaftlichen Datenanalyse, S. 827-836} (Thulb E-Book)}
\item{Statistik: Probleme mit dem stat. Testen,\textit{Statistik: Eine Einführung für Sozialwissenschaftler,
Ludwig-Mayerhofer  } (muss kopiert werden)}
\item{\texttt{STATA}: \textit{Kohler-Kreuter,Datenanalyse mit Stata: Allgemeine Konzepte der Datenanalyse und ihre praktische Anwendung}, Kap I}
\item{\texttt{STATA}: Regression:\\  \url{stats.idre.ucla.edu/stata/webbooks/reg/chapter1/}\\ \url{regressionwith-statachapter-1-simple-and-multiple-regression/}}
\item{\texttt{STATA}: logistische Regression: \url{stats.idre.ucla.edu/stata/dae/logistic-regression/}}
\end{enumerate}

% STATA:
% Das erste Mal
% dann Links
% https://stats.idre.ucla.edu/stata/dae/logistic-regression/
% https://stats.idre.ucla.edu/stata/webbooks/reg/chapter1/regressionwith-statachapter-1-simple-and-multiple-regression/

% Statistik:
% neues Buch von Eid
% Auch in Stat II
% Einfach Effektstärke noch einführen und
% Teile als Formelsammlung kopieren.
%    \begin{table}
%  %  \caption{Mit zwei Ausnahmen findet die Vorlesung $14$-tägig statt}
%\begin{center}
%\caption{\colorbox{yellow!40}{Es gibt ein Aufgabenblatt zur Abgabe}}
%  \begin{tabular}{|c|c|c|}
%    \hline
%    Nr. & Datum & Thema \\ \hline
%    0 &   $ 16.10.2017$   &  \colorbox{blue!10}{  Der Zufallsvorgang und der Inferenzschluss}   \\ \hline
%    \colorbox{yellow!40}{1} &  $23.10.2017$  &   \colorbox{blue!10}{Die parametrische  Wahrscheinlichkeitsverteilung } \\ \hline
%    \colorbox{yellow!40}{2} &   $13.11.2017$   &   \colorbox{green!40}{Grundlagen des statistischen Testens}\\ \hline
%    \colorbox{yellow!40}{3} &   $27.11.2017$   &   \colorbox{green!40}{Punktschätzer und Konfidenzintervalle }\\ \hline
%%    \multirow{2}{*}{3} &  \multirow{2}{*}{$27.11.2017$}& \colorbox{red!20}{Grundlagen des statistischen Testens}\\
%%               &               & von quantitativen Merkmalen\\ \hline
%  \multirow{2}{*}{\colorbox{yellow!40}{4}} &  \multirow{2}{*}{$11.12.2017$}& \colorbox{violet!40}{Bivariate lineare Regression} \\
%                   &        & \colorbox{violet!40}{und Varianzanalyse}\\ \hline  
%    \colorbox{yellow!40}{5} & $08.01.2018$      &  \colorbox{violet!40}{Multiple lineare Regression}  \\ \hline   
%    6 &   $22.01.2018$     &  \colorbox{violet!40}{Varianzanalyse} \\ \hline
%  \end{tabular}
%  \end{center}
%  \label{tab:multicol}
%  \end{table}
\section{Ablaufplan für dieses Semester}
Die folgende Terminübersicht zeigt jeweils den Schwerpunkt der entsprechenden Seminarübung. Wenn der Schwerpunkt im Bereich
Statistik liegt, kann auch teilweise mit \texttt{STATA} gearbeitet werden. Der Schwerpunkt ist also das Hauptthema des entsprechenden
Termins.
\begin{table}[h]
  %  \caption{Mit zwei Ausnahmen findet die Vorlesung $14$-tägig statt}
\begin{center}
\caption{Überblick über die Veranstaltungen in 2017}
  \begin{tabular}{|c|c|c|}
    \hline
    Nr. & Datum & Thema \\ \hline
   
    0 &   $18.10.2017$   &    Organisatorisches, Arbeitsweise, Ablauf, Vorstellung der Materialien  \\ \hline
     1 &  $25.10.2017$     &  \colorbox{green!50}{Artikel: Worum geht es?} \\ \hline
    2 &   $01.11.2017$   &   \colorbox{yellow!60}{ Der Zufallsvorgang und der Inferenzschluss }  \\ \hline
    3 &   $08.11.2017$  &    \colorbox{yellow!60}{Grundlagen des statistischen Testens}\\ \hline
    4 &   $15.11.2017$   &  \colorbox{yellow!60}{Das lineare Regressionsmodell } \\ \hline
    5 &   $22.11.2017$     & \colorbox{green!50}{Artikel: Eigene Fragestellung vorstellen} \colorbox{red!70}{1. PL} \\ \hline
    6 &   $29.11.2017$   &   \colorbox{pink}{\texttt{STATA}: Das erst Mal (Kohler und Kreuter, Kap I)}\\ \hline
%    \multirow{2}{*}{3} &  \multirow{2}{*}{$27.11.2017$}& \colorbox{red!20}{Grundlagen des statistischen Testens}\\
%               &               & von quantitativen Merkmalen\\ \hline
  \multirow{2}{*}{7} &  \multirow{2}{*}{$06.12.2017$}& \colorbox{pink}{\texttt{STATA}: Hypothesen operationalisieren
  } \\
                   &        & \colorbox{pink}{Variablen auswählen und geeignet codieren}\\ \hline  
  %  6 & $22.11.2018$      &  \colorbox{pink}{Regression mit \texttt{STATA} I }  \\ \hline   % Link
    8 & $13.12.2017$      &  \colorbox{pink}{Das lineare Regressionsmodell  mit \texttt{STATA} }  \\ \hline   % Link
%    7 & $06.12.2017$     &  \colorbox{green!50}{Artikel: Worum geht es?} \\ \hline
    9 & $20.12.2017$     &  \colorbox{yellow!60}{Zustände und Ereignisse in den Sozialwissenschaften +GLM (Windzio)}  \\ \hline
  %  9 & $20.12.2017$     & \colorbox{green!50}{Artikel: Eigene Fragestellung ableiten} \\ \hline
  %  11 & $$     &   \\ \hline
  \end{tabular}
  \end{center}
  \label{tab:multicol}
 \end{table}
 
 \begin{table}[h]
  %  \caption{Mit zwei Ausnahmen findet die Vorlesung $14$-tägig statt}
\begin{center}
\caption{Überblick über die Veranstaltungen in 2018}
  \begin{tabular}{|c|c|c|}
    \hline
    Nr. & Datum & Thema \\ \hline
    10 &   $10.01.2018$   &   \colorbox{yellow!60}{ Das logistische Regressionsmodell I (Windzio)} \\ \hline
    11 &   $17.01.2018$   &   \colorbox{pink}{ Das logistische Regressionsmodell mit \texttt{STATA}}  \\ \hline
    12 &   $24.01.2018$  &   \colorbox{yellow!60}{ Das logistische Regressionsmodell II (Windzio)}\\ \hline
    13 &   $31.01.2018$   &   \colorbox{green!50}{Artikel: Präsentation und Diskussion der eigenen Daten-Analyse I} \colorbox{red!70}{2. PL}\\ \hline
    14 &   $07.02.2018$   &    \colorbox{green!50}{Artikel: Präsentation und Diskussion der eigenen Daten-Analyse II} \colorbox{red!70}{2. PL}\\ \hline
\end{tabular}
  \end{center}
  \label{tab:multicol}
 \end{table}
\section{Erbringen der Prüfungsleistung (PL)}
Die Zulassung zur Prüfung erfolgt durch die Abgabe der ersten Prüfungsleistung (in einer annehmbaren Form).
Die Prüfungsleistung besteht insgesamt aus drei Teilen.\\
 Wählen Sie einen der beiden Artikel aus und bearbeiten Sie die folgenden
Aufgaben:
\begin{itemize}
\item[1)]{Der erste Teil der Prüfungsleistung besteht darin, den gewählten  Artikel zusammenzufassen und eine eigene Hypothese
abzuleiten, die man mittels einer (logistischen) Regression untersuchen kann.
 Wichtige  Aussagen des gewählten  Artikel sollen zunächst auf zwei bis drei Seiten zusammengefasst werden. 
 Die Ableitung der eigenen Hypothese sollte auf einer halben bis maximal einer ganzen Seite dargelegt werden.
Auf einer halben bis maximaler einer Seite soll der Artikel mit Bezug auf die eigene Hypothese  kritisch und konstruktiv kommentiert werden.
\\
Bearbeitungszeit ist vom 18.10.2017 bis zum 17.11.2017 um 18 Uhr, bitte per E-Mail an mich abgeben. Am 22.11.2017 werden
die Hypothesen im Seminar vorgestellt. Es erfolgt eine Rückmeldung, ob die Hypothese für die weitere Arbeit geeignet ist.}
\item[2)]{Der zweite Teil der Prüfungsleistung besteht darin, mit Bezug auf den Artikel die eigne Hypothese,
mit einem (logistischen) Regressionsmodell zu untersuchen. 
\\ Stellen
Sie mit Bezug zu dem Artikel ein eigenes (logistisches) Regressionsmodell  auf, rechnen Sie es mit \texttt{STATA}
und interpretieren Sie das Ergebnis. Der Bearbeitungszeitraum ist vom 6.12.2017 bis zum 19.01.2018 um 18 Uhr, bitte per E-Mail an mich abgeben.
Am 31.01.2018 und am 7.02.2018 werden die Ergebnisse im Seminar vorgestellt und besprochen.\\
Für den zweiten Teil der Prüfungsleistung wird ihnen ein Datensatz zur Verfügung gestellt, der dem KIGGS-Datensatz
ähnelt. Mit ihren \texttt{do} Files werde ich die Berechnung auf dem echten KIGGS-Datensatz durchführen und Ihnen
die Ergebnisse mitteilen. }
\item[3)]{Der dritte Teil der Prüfungsleistung besteht darin den ersten und zweiten Teil in einem Forschungsbericht zusammenzufassen.
Dieser sollte insgesamt etwa acht bis zehn Seiten lang sein. Abgabetermin ist der 23.03.2018. Auch der \texttt{STATA do} File
soll abgegeben werden.}
\end{itemize}

%\section{Anmerkungen}
%\begin{itemize}
%\item{Beide Artikel bekommen die Studis von Anfang an und sie sollen auch von Anfang versuchen, die Artikel zu lesen.}
%\item{Bei den grün markierten Seminaren, arbeiten wir im Seminar gemeinsam an Fragestellungen, die in
%Bezug auf die Hausarbeit entstehen.}
%\item{Zu den gelben Seminaren gibt es Vorlesungsfolien, außerdem wird die Formelsammlung verwendet.}
%\item{Zu den \texttt{STATA}-Seminaren gibt es \texttt{do}-Files.}
%\item{Ich möchte die Beispiele und Daten aus dem Buch von Michael Windzio verwenden und ergänzen.
%Diese Aufgabe könnte ich dem Praktikanten übertragen. Er kann die \texttt{STATA}-Seminare mit vorbereiten.}
%\item{Für die Bearbeitung der eigenen Regressionsaufgabe bekommen die Studies einen Subdatensatz (sie müssen ihn im
%Methoden Pool bearbeiten)}
%\end{itemize}
%
%\section{Fragen zu den Artikeln}
%\begin{enumerate}
%\item{Hat Frau Krause lauter einzelne Regressionen gerechnet, mit je einer Einflussgröße und jeweils das $R^{2}$ angegeben?}
%\item{Wie sieht das Modell aus, dass Marcel gerechnet hat, einfach ein lineares Wahrscheinlichkeitsmodell?}
%\end{enumerate}
\end{document}

